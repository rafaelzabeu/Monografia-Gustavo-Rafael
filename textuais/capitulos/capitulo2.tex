\chapter{Aprendizado por Reforço}
\label{chap:reinf}

 Este capítulo apresenta o conceito de aprendizado por reforço, uma técnica de aprendizado onde não se informa qual a saída desejada, mas procura maximizar o reforço das ações boas de um sistema.
 
\section{Redes Neurais}
\label{sec:neural}

Apresenta o conceito de redes neurais, uma técnica de inteligência artificial que procura imitar o funcionamento do cérebro, tendo neurônios artificiais que se conectam por pesos.

\subsection{Regressão}
\label{subsec:regression}

Nesta seção é dada a definição de regressão, como a técnica é utilizada para analisar relações de dados e uma explicação breve da diferença entre regressão linear e regressão não linear.

\subsection{Perceptron}
\label{subsec:perceptron}

A seção falará sobre o conceito básico de um perceptron, que é classificar entradas, sua composição e demonstra seu funcionamento.

\subsection{Deep Learning}
\label{subsec:deep}

Nesta seção é apresentado o conceito de aprendizado profundo, que utiliza redes neurais massivas com várias camadas.

\section{Algoritmo Genético}
\label{sec:genetic}

Esta seção descreve o algoritmo genético, um paradigma derivado do modelo evolucionário, onde elementos de classificação competem entre si para classificar uma entrada, elementos com performance fraca são descartados e os com performance forte proliferam.

%Com a avanco da tecnologia da informacao e com o aumente da quantidade de dados sendo produzido atualmete bibliotecas como MapReduce, Hadoop e Spark acabam por tornar o gerenciamente desse big data possivel. Transformando muito do que seria complexo de se gerenciar em funcoes de alto nivel simples como map e reduce, fazendo com essas operacoes sejam executadas em cluster de forma paralela com o maximo de abstração de onde esta se executando.
