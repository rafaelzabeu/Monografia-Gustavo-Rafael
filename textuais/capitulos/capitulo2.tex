\chapter{Aprendizado por Reforço}
\label{chap:reinf}

 Este capítulo apresenta o conceito de aprendizado por reforço, uma técnica de aprendizado onde não se informa qual a saída desejada, mas procura maximizar o reforço das ações boas da rede.
 
\section{Redes Neurais}
\label{sec:neural}


\subsection{Regressão}
\label{subsec:regression}


\subsection{Perceptron}
\label{subsec:perceptron}


\subsection{Deep Learning}
\label{subsec:deep}


\section{Algoritmo Genético}
\label{sec:genetic}



%Com a avanco da tecnologia da informacao e com o aumente da quantidade de dados sendo produzido atualmete bibliotecas como MapReduce, Hadoop e Spark acabam por tornar o gerenciamente desse big data possivel. Transformando muito do que seria complexo de se gerenciar em funcoes de alto nivel simples como map e reduce, fazendo com essas operacoes sejam executadas em cluster de forma paralela com o maximo de abstração de onde esta se executando.
