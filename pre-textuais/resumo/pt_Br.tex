\setlength{\absparsep}{18pt} % ajusta o espaçamento dos parágrafos do resumo
\begin{resumo}
O objetivo deste projeto é estudar três estratégias diferentes de aprendizado por reforço com inteligência artificial para gerenciamento de recursos e conflitos dentro do jogo Starcraft II. 

Trabalharemos com redes neurais, redes neurais com algoritmo genético e um modelo similar ao utilizado no projeto AlphaGO. Ferramentas para a integração com o jogo Starcraft II foram criadas pela Blizzard(criadora do jogo) em parceria com a Deep Mind(Google). 

Uma inteligência artificial que entenda e resolva os problemas no cenário do Starcraft II pode ser aplicada em situações do mundo real. Por exemplo, lidar com priorização de sistemas críticos, gerenciamento de conflitos em crises de paz e tomada de decisões em sistemas estratégicos, visto que este é um jogo de estratégia no qual os principais objetivos é o gerenciamento de recurso e de conflito.


 \textbf{Palavras-chave}: Inteligência Artificial. Starcraft II. Aprendizado por Reforço. Jogos.
\end{resumo}